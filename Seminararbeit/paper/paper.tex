\documentclass[conference]{IEEEtran}
\IEEEoverridecommandlockouts
% The preceding line is only needed to identify funding in the first footnote. If that is unneeded, please comment it out.
\usepackage{cite}
\usepackage{amsmath,amssymb,amsfonts}
\usepackage{algorithmic}
\usepackage{graphicx}
\usepackage{textcomp}
\usepackage{xcolor}
\def\BibTeX{{\rm B\kern-.05em{\sc i\kern-.025em b}\kern-.08em
    T\kern-.1667em\lower.7ex\hbox{E}\kern-.125emX}}
\begin{document}

\title{Vergleich verlustfreier Datenkompressionsverfahren auf Bilddaten}

\author{
    \IEEEauthorblockN{Nick Schreiber}
    \IEEEauthorblockA{Technische Hochschule Rosenheim\\
    Master Informatik, Seminar theoretische Informatik\\
    Email: nick.schreiber@stud.th-rosenheim.de}
}

\maketitle

\begin{abstract}
This document is a model and instructions for \LaTeX.
This and the IEEEtran.cls file define the components of your paper [title, text, heads, etc.]. *CRITICAL: Do Not Use Symbols, Special Characters, Footnotes, 
or Math in Paper Title or Abstract.
\end{abstract}


% \begin{IEEEkeywords}
% component, formatting, style, styling, insert
% \end{IEEEkeywords}

\section{Einleitung}

Datenkompression beschreibt ein Verfahren, das zum Ziel hat, eine Nachricht 
ohne relevanten Informationsverlust zu verkleinern.  
Als Nachricht ist jede Art von digitalen Daten gemeint, z.B. Text, Bild, Audio, etc..   
Daten können komprimiert werden, indem Redundanz entfernt oder eine Kodierung angewendet wird.
Daher wird Datenkompression oft als Kodierung bezeichnet.
Kodierung ist ein allgemeiner Begriff, der jede spezielle Darstellung von Daten nach 
einem bestimmten Schema umfasst. \cite{Ingles}

Es gibt zwei Arten der Datenkompression: die verlustbehaftete und die verlustfreie Kompression. 
Bei der verlustbehafteten Datenkompression kann eine bestimmte Menge an Information durch die 
Kompression verloren gehen, was in Kauf genommen wird, da dadurch die Datenmenge erheblich 
reduziert werden kann oder weil die verlorene Informationen für die Anwendung kaum relevant sind.
Das wird auch als Irrelevanzreduktion bezeichnet \cite[S. 5]{Maluck}.
Ein Beispiel für Irrelevanzreduktion kann bei Audiosignalen beobachtet werden. 
Der menschliche Hörfrequenzbereich liegt zwischen 20 Hz und 20 kHz \cite{Burke}. 
Daher ist es nicht sinnvoll, Frequenzen, die weit außerhalb des hörbaren Bereichs liegen, 
in Audiodateien zu speichern. 
Bei der verlustfreien Datenkompression wird die Integrität der Daten bewahrt.
Das bedeutet, dass sämtliche Informationen in den komprimierten Daten enthalten sind 
und die Originaldaten vollständig rekonstruierbar sind.
In dieser Arbeit wird nur die verlustfreie Datenkompression untersucht, da 
Irrelevanzreduktion nicht direkt zum Themengebiet der Datenkompression gehört.

Die Datenkompression von Bildern wird aus verschiedenen Gründen eingesetzt. 
Speichernutzung: Unkomprimierte Bilddaten können beträchtlich mehr Speicherplatz beanspruchen. 
Übertragungseffizienz: Bei der Übertragung von Bildern über Netzwerke oder das Internet spielt 
die Übertragungseffizienz eine entscheidende Rolle. 
Wenn ein Bild über einen Kanal mit begrenzter Bandbreite gesendet wird, kann es effizienter 
sein, das Bild zu komprimieren, es zu übertragen und dann beim Empfänger zu dekomprimieren. 
Dadurch wird die Übertragungszeit verkürzt und das Bild kann schneller bereitgestellt werden.
Dies führt zu einer höheren Übertragungsrate und einer reduzierten Bandbreitennutzung.



\section{Zielsetzung der Arbeit}

Ziel der Arbeit ist es zu untersuchen, ob und warum bestimmte verlustfreie 
Datenkompressionsverfahren für Bilddaten besser geeignet sind als andere.
Dazu werden die theoretischen Aspekte der Kompressionsalgorithmen untersucht.
Außerdem wird untersucht, wie Bilddaten aufgebaut sind und welche Besonderheiten 
in der Datenstruktur für die Datenkompression genutzt werden können.
Die Arbeit beinhaltet auch einen praktischen Teil.
Verschiedene Algorithmen zur verlustfreien Datenkompression wurden manuell 
implementiert und an unterschiedlichen Bilddaten getestet.
So konnten konkrete Ergebnisse über die Leistungsfähigkeit der Algorithmen gewonnen werden.
Die verglichenen Algorithmen sind Run Length Encoding (RLE), Huffman Encoding, Lempel-Ziv 1977 (LZ77), 
PNG Algorithmus und verschiedene Kombinationen der Algorithmen. 
Die Ergebnisse werden interpretiert und mit den theoretischen Erwartungswerten verglichen.






\section{Introduction}
This document is a model and instructions for \LaTeX.
Please observe the conference page \cite{autor2023} limits. 

\section{Ease of Use}

\subsection{Maintaining the Integrity of the Specifications}

The IEEEtran class file is used to format your paper and style the text. All margins, 
column widths, line spaces, and text fonts are prescribed; please do not 
alter them. You may note peculiarities. For example, the head margin
measures proportionately more than is customary. This measurement 
and others are deliberate, using specifications that anticipate your paper 
as one part of the entire proceedings, and not as an independent document. 
Please do not revise any of the current designations.


\bibliographystyle{IEEEtran}
\bibliography{mybib}

\end{document}
