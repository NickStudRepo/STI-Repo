\documentclass[conference]{IEEEtran}
\IEEEoverridecommandlockouts
% The preceding line is only needed to identify funding in the first footnote. If that is unneeded, please comment it out.
\usepackage{cite}
\usepackage{amsmath,amssymb,amsfonts}
\usepackage{algorithmic}
\usepackage{graphicx}
\usepackage{textcomp}
\usepackage{xcolor}
\def\BibTeX{{\rm B\kern-.05em{\sc i\kern-.025em b}\kern-.08em
    T\kern-.1667em\lower.7ex\hbox{E}\kern-.125emX}}
\begin{document}

\title{Vergleich verlustfreier Datenkompressionsverfahren auf Bilddaten}

\author{
    \IEEEauthorblockN{Nick Schreiber}
    \IEEEauthorblockA{Technische Hochschule Rosenheim\\
        Master Informatik, Seminar theoretische Informatik\\
        Email: nick.schreiber@stud.th-rosenheim.de}
}

\maketitle

\begin{abstract}
    This document is a model and instructions for \LaTeX.
    This and the IEEEtran.cls file define the components of your paper [title, text, heads, etc.]. *CRITICAL: Do Not Use Symbols, Special Characters, Footnotes,
    or Math in Paper Title or Abstract.
\end{abstract}


% \begin{IEEEkeywords}
% component, formatting, style, styling, insert
% \end{IEEEkeywords}

\section{Einleitung}

Datenkompression beschreibt ein Verfahren, das zum Ziel hat, eine Nachricht
ohne relevanten Informationsverlust zu verkleinern.
Als Nachricht ist jede Art von digitalen Daten gemeint, z.B. Text, Bild, Audio, etc..
Daten können komprimiert werden, indem Redundanz entfernt oder eine Kodierung angewendet wird.
Daher wird Datenkompression oft als Kodierung bezeichnet.
Kodierung ist ein allgemeiner Begriff, der jede spezielle Darstellung von Daten nach
einem bestimmten Schema umfasst. \cite{Ingles}

Es gibt zwei Arten der Datenkompression: die verlustbehaftete und die verlustfreie Kompression.
Bei der verlustbehafteten Datenkompression kann eine bestimmte Menge an Information durch die
Kompression verloren gehen, was in Kauf genommen wird, da dadurch die Datenmenge erheblich
reduziert werden kann oder weil die verlorene Informationen für die Anwendung kaum relevant sind.
Das wird auch als Irrelevanzreduktion bezeichnet \cite[S. 5]{Maluck}.
Ein Beispiel für Irrelevanzreduktion kann bei Audiosignalen beobachtet werden.
Der menschliche Hörfrequenzbereich liegt zwischen 20 Hz und 20 kHz \cite{Burke}.
Daher ist es nicht sinnvoll, Frequenzen, die weit außerhalb des hörbaren Bereichs liegen,
in Audiodateien zu speichern.
Bei der verlustfreien Datenkompression wird die Integrität der Daten bewahrt.
Das bedeutet, dass sämtliche Informationen in den komprimierten Daten enthalten sind
und die Originaldaten vollständig rekonstruierbar sind.
In dieser Arbeit wird nur die verlustfreie Datenkompression untersucht, da
Irrelevanzreduktion nicht direkt zum Themengebiet der Datenkompression gehört.

Die Datenkompression von Bildern wird aus verschiedenen Gründen eingesetzt.
Speichernutzung: Unkomprimierte Bilddaten können beträchtlich mehr Speicherplatz beanspruchen.
Übertragungseffizienz: Bei der Übertragung von Bildern über Netzwerke oder das Internet spielt
die Übertragungseffizienz eine entscheidende Rolle.
Wenn ein Bild über einen Kanal mit begrenzter Bandbreite gesendet wird, kann es effizienter
sein, das Bild zu komprimieren, es zu übertragen und dann beim Empfänger zu dekomprimieren.
Dadurch wird die Übertragungszeit verkürzt und das Bild kann schneller bereitgestellt werden.
Dies führt zu einer höheren Übertragungsrate und einer reduzierten Bandbreitennutzung.



\section{Zielsetzung der Arbeit}

Ziel der Arbeit ist es zu untersuchen, ob und warum bestimmte verlustfreie
Datenkompressionsverfahren für Bilddaten besser geeignet sind als andere.
Dazu werden die theoretischen Aspekte der Kompressionsalgorithmen untersucht.
Außerdem wird untersucht, wie Bilddaten aufgebaut sind und welche Besonderheiten
in den Bilddaten für die Datenkompression genutzt werden können.
Die Arbeit hat einen praktischen Anteil.
Verschiedene Algorithmen zur verlustfreien Datenkompression wurden manuell
implementiert und an unterschiedlichen Bilddaten getestet.
So konnten konkrete Ergebnisse über die Leistungsfähigkeit der Algorithmen gewonnen werden.
Die verglichenen Algorithmen sind Run Length Encoding (RLE), Huffman Encoding, 
Lempel-Ziv 1977 (LZ77), PNG Algorithmus und verschiedene Kombinationen der Algorithmen.
Die Ergebnisse werden interpretiert und mit den theoretischen Erwartungswerten verglichen.

% Todo ab hier nochmal verbessern!
\section{Grundlagen zur Datenkompression}

In der Informatik gehört die Datenkompression zum Teilgebiet der Informationstheorie.
Um zu verstehen, wie verlustfreie Datenkompression funktioniert, muss man einige theoretische
Grundlagen kennen.

\subsection{Information}

Claude Shannon, der Erfinder der Informationstheorie, definiert den Begriff Information
als Maß für den Informationsgehlt.
Information ist ein Maß der Unsicherheit, das durch das Eintreten eines bestimmten
Ereignisses oder das Empfangen einer Nachricht verringert wird. \cite{shannon}
Information ist die Mindestanzahl von Bits, die zur Codierung einer Nachricht verwendet
werden müssen. \cite{shannon2}
Die grundlegende Idee zu Information besteht darin, dass Informationen umso wertvoller sind,
je unerwarteter oder unwahrscheinlicher sie sind.

\subsection{Entropie}

Die Quantifizierung des Informationsgehalts erfolgt durch die Entropie ($H$).
Formal drückt die Entropie die durchschnittliche Menge an Bits aus,
die benötigt werden, um eine Information zu kodieren. \cite{shannon}
Die Entropie berücksichtigt die Wahrscheinlichkeiten verschiedener möglicher
Ereignisse und erreicht ein Maximum, wenn alle Ereignisse gleich wahrscheinlich sind,
was auf maximale Unsicherheit hinweist.

\begin{equation}
    \label{eq:entropie}
    H(X) = -\sum_{i=1}^{n} P(x_i) \cdot \log_{2}(P(x_i))
\end{equation}

Formel \ref{eq:entropie} definiert die Entropie mathematisch.
Hierbei steht $H(X)$ für die Entropie der Menge $X$.
$P(x_i)$ steht für die Wahrscheinlichkeit des
Auftretens des Ereignisses $x_i$.
Die Summe wird über alle möglichen Ereignisse
$x_i$ in $X$ gebildet.

Diese Formel beschreibt die durchschnittliche Menge an Bits, die benötigt werden,
um eine Nachricht aus X zu kodieren.
Wenn die Entropie hoch ist, ist die Unsicherheit groß, und es werden mehr Bits
benötigt, um die Informationen zu repräsentieren.
Wenn die Entropie niedrig ist, gibt es weniger Unsicherheit, und somit werden
weniger Bits benötigt.

Man kann nun eine direkte Verbindung zwischen Entropie und Kompression herzustellen.
Niedrige Entropie bedeutet, dass eine Datenmenge strukturiert ist, bzw. Muster aufweist.
Das heißt, dass in den Daten wenig unsicherheit ist und die Daten Redundanz enthalten.
Das bedeutet, niedrige Entropie sagt, dass die Daten komprimiert werden können.


\subsection{Redundanz und Mutual Information}

Redundanz beschreibt Informationen die in Daten mehrfach vorhanden sind. \cite{friedrichs}
Einfach gesagt kann man Redundanz als überflüssige Information betrachten.
Eine hohe Redundanz sagt aus, dass sich wiederholende oder vorhersehbare Muster innerhalb
der Daten befinden.

Um eine Formel für die Redundanz aufzustellen benötigt man die mittlere Codewortlänge.
Die mittlere Codewortlänge gibt den durchschnittlichen Bedarf an Bits pro Symbol in einer
Nachricht an.
Sei $X$ ein Alphabet und $x \in X$.
$C(x)$ bezeichnet das zu $x$ gehörende Codewort.
$l(x)$ bezeichnet die Länge von $C(x)$.
Die mittlere Codewortlänge L(C) einer Nachricht C(x) mit der Wahrscheinlichkeitsverteilung
p(x) ist in Formel \ref{eq:codewortlänge} definiert.

\begin{equation}
    \label{eq:codewortlänge}
    L(C) = \sum_{i}^{|X|} p(x_i) \cdot l(x_i)
\end{equation}

Mit der mittleren Codewortlänge lässt sich nun die Redundanz des Codes, bzw. der 
Nachricht berechnen. 
Die Formel \ref{eq:redundanz} definiert die Redundanz einer Nachricht.

\begin{equation}
    \label{eq:redundanz}
    R_{\text{Code}} = L(C) - H(X)
\end{equation}

Die Redundanz wird berechnet, indem von der tatsächlichen durchschnittlichen 
Anzahl an Bits pro Symbol die theoretisch minimale Anzahl an Bits pro Symbol 
abgezogen werden. 
Die theoretisch minimale Anzahl an Bits pro Symbol entspricht der enthaltenen 
Information und ist gleich der Entropie der Nachricht.
Daraus ergibt sich, dass die Redundanz $\geq$ 0 sein muss. 

Mutual Information ist ein quantitatives Maß für die gegenseitige Abhängigkeit von zwei
Variablen. \cite{shannon}
Es misst, wie sehr die Kenntnis einer Variablen die Unsicherheit über die andere
Variable reduziert.
Dieses Konzept ist entscheidend, um die Struktur von Daten zu verstehen und
voneinander abhängige Informationen zu erkennen.

Wenn die Mutual Information zwischen zwei Variablen hoch ist, bedeutet dies,
dass das Wissen über eine Variable
bedeutende Informationen über die andere Variable liefert.
Hohe Mutual Information sagt dementsprechend aus, dass zwei Variablen stark voneinander
Abhängig sind.
Das Wissen über den Wert einer Variable trägt bereits wesentlich zur Vorhersage
oder zum Verständnis der anderen Variable bei.

Geringe Mutual Informaion sagt aus, dass die beiden Variablen weniger gemeinsame
Information teilen.
Das Wissen über den Wert einer Variable trägt nicht so stark zur
Vorhersage oder zum Verständnis der anderen Variable bei.
Das bedeutet eine schwächere Statistische Abhängig der Variablen.

Durch das erkennen von Mutual Information kann gezeigt werden, dass Muster und/ oder
Wiederholungen und dementsprechend Redundanz in den Daten enthalten ist.
Redundanz spielt im Bezug auf Kompression eine wichtige Rolle.
Kompression funktioniert durch das identifizieren und eliminieren redundanter Elemente,
um den Informationsgehalt zu maximieren und die Effizienz von Datenrepräsentationen
zu steigern.


\section{Informationstheorie}

Eine wichtige Punkt in der Informationstheorie ist die Unterscheidung 
zwischen Daten und Information.
Daten und Information werden im normalen Sprachgebrauch häufig als Synonym verwendet,
was eigentlich nicht korrekt ist.

Daten sind rohe Fakten oder Symbole, die an sich keine spezifische Bedeutung haben.
Informationen entstehen durch die Interpretation, Organisation und Strukturierung
von Daten, wodurch ein sinnvoller Kontext geschaffen wird. \cite{pieper}
Daten werden zu Informationen, wenn sie für einen bestimmten Zweck verwendet werden können.

Im Kontext der Datenkompression ist es wichtig zu verstehen, dass nicht alle Daten
gleichermaßen informativ sind. 
Ein effektiver Kompressionsalgorithmen entfernt redundante
und nicht informative Teile der Daten.
Jedoch bleibt die gesamte Information enthalten. 
Die Daten sind somit invormativer und komprimierter als zuvor.

\subsection{Quellencodierungstheorem/ Source Coding Theorem}

Das Quellencodierungstheorem beschäftigt sich mit der Effizienz der Datenkompression
und sagt aus, dass es eine Grenze für die minimale mittlere Codierungslänge gibt, 
die für die Darstellung von Information aus einer bestimmten Quelle erforderlich 
ist. \cite{sharma} 
Das Quellencodierungstheorem besagt, dass die mittlere Codierungslänge $L$ pro Symbol 
für eine gegebene Quelle nicht kleiner sein kann als die Entropie $H$ der Quelle. 
Mathematisch in Formel \ref{eq:sct} ausgedrückt.

\begin{equation}
    \label{eq:sct}
    L \ge H
\end{equation}

Die Entropie stellt dabei die untere Schranke für die mittlere Codierungslänge dar.
Das zeigt, dass Datenkompression nicht bis ins unendliche möglich ist ohne 
Information zu verlieren. 
Die Maximale Kompression ist genau dann erreicht, wenn $L = H$ entspricht.
Es würde also bei solchen Daten keinen sinn machen zu versuchen die Daten 
zu komprimieren, da das ohne Informationsverlust nicht möglich ist. 

Vlt. Todo: Schubfachprinzip, Bedeutung von Entropie in diesem Kontext.

\subsection{Kolmogorov Komplexität}

Die Kolmogorov Komplexität ist ein Maß für die Strukturiertheit einer Zeichenkette. 
Sie entspricht der Länge des kürzesten Programms, das die Zeichenkette erzeugen 
kann. \cite{li}
Die Kolmogorov Komplexität ist dementsprechend ein Maß für die algorithmische Komplexität 
von Informationen.

Die Kolmogorov Komplexität eines Objekts, z. B. eines Textes, ist die Länge 
des kürzesten Programms, das das Objekt als Ausgabe erzeugt.
Hier ein Beispiel für so ein Programm.
Betrachten wir die folgende Zeichenkette: "AAAAAAAAABBBBBCCCCCC".
Die Zeichenkette besteht aus 20 Zeichen (9 x A, 5 x B, 6 x C).
Mit einem einfachen Programm lässt sich die Zeichenkette deutlich kürzer
beschreiben: "9A5B6C".
Das Programm gibt die Länge der identischen aufeinanderfolgenden Zeichen an, 
gefolgt von dem Zeichen.
So lässt sich die ursprüngliche Zeichenkette der Länge 20 in nur 6 Zeichen darstellen.
Wir haben ein Programm gefunden, das die Länge der Beschreibung der Zeichenkette 
erheblich reduziert.
Die Kolmogorov Komplexität dieser Zeichenkette ist deutlich geringer als
die ursprüngliche Länge der Zeichenkette.
Das zeigt, dass in der Zeichenkette Strukturen vorhanden sind.

Es ist wichtig zu beachten, dass die tatsächliche Kolmogorov Komplexität für 
allgemeine Zeichenketten wegen des Halteproblemes nicht praktisch 
berechenbar ist. \cite{OPPaper}
Allerdings können Abschätzungen gemacht werden. 
Wenn ein Algorithmus gefunden wird, der eine Zeichenkette in einem kürzeren 
Programm darstellt entspricht die Kolmogorov Komplexität der Zeichenkette 
maximal der Länge des Programms.
Es ermöglicht den Informationsgehalt von Daten in
Bezug auf die kürzeste mögliche algorithmische Beschreibung zu verstehen.

% Eine enge Verbindung besteht zur Vorstellung
% der optimalen, universellen Datenkompression – je einfacher die Beschreibung, desto
% effizienter kann die Datenmenge komprimiert werden.
% Dieses Konzept hat weitreichende praktische Anwendungen, von der Identifikation von
% wiederholten Mustern bis hin zur Bewertung von Algorithmeneffizienz.
% Allerdings stehen der Anwendung auch Herausforderungen gegenüber, wie der Unberechenbarkeit
% des absoluten Komplexitätsmaßes und der Schwierigkeit, universelle Algorithmen zu finden,
% die für alle Daten gleichermaßen effizient sind. Das Verständnis der Kolmogorov-Komplexität
% trägt maßgeblich zur Entwicklung fortgeschrittener Datenkompressionsverfahren bei und bietet
% Einblicke in die Grenzen der Komprimierbarkeit von Informationen.

% Anforderungen an Daten, damit diese komprimierbar sind
\subsection{Datenanforderungen, Komprimierbarkeit}

Daten können in verschiedene Gruppen eingeteilt werden.
Auf der einen Seite stehen strukturierte Daten und auf der anderen Seite
unstrukturierte Daten.

Strukturierte Daten sind Daten in denen wiederkehrende oder vorhersagbare Muster 
enthalten sind. 
Die Entropie der Daten ist niedrig.
Ein Beispiel für strukturierte Daten sind Tabellen. 

Unstrukturierte Daten sind Daten, in denen keine wiederkehrende oder vorhersagbare 
Muster enthalten sind.
Die Entropie der Daten ist hoch.
Ein Beispiel für unstrukturierte Daten sind zufällig erzeugte Daten.

Strukturierte Daten enthalten meist eine höhere Redundanz als unstrukturierte 
Daten. 
Das ist dem Fakt geschuldet, dass die maximal mögliche Kompression, der Information in den 
Daten, der Entropie, entspricht. 
Daten können theoretisch maximal auf das Niveau der Entropie der Daten komprimiert werden.
Strukturierte Daten mit geringer Entropie haben ein tieferes Limit als unstrukturierte Daten.
Deshalb können strukturierte Daten meist mehr komprimiert werden als unstrukturierte.

\subsubsection{Vorverarbeitung}

Eine Möglichkeit aus etwas unstrukturierten Daten strukturiertere zu machen ist über 
eine Vorverarbeitung der Daten.
So eine Vorverarbeitung ist meist eine Normalisierung oder eine Datenfilterung.
Wichtig ist, dass der Vorverarbeitungsschritt umkehrbar ist.
Eine Art wie Daten für die Kompression Vorverarbeitet werden, wird im Lauf der 
Arbeit anhand des PNG Algorithmus gezeigt. 
PNG verwendet eine Datenfilterung.

\subsubsection{Limitationen}

Ein Beispiel für maximal unstrukturierte Daten die scheinbar nicht komprimierbar 
sind, sind normalverteilte Zufallszahlen.
"The Random Compression Challenge" von Mark Nelson \cite{nelson} betrachtet genau dieses
Problem. 
Das Ziel der Challenge ist es eine Datei, die etwa ein halbes Megabyte groß ist 
zu komprimieren. 
Die Datei besteht aus einer Millionen Zufallszahlen, die gleichverteilt sind und 
aus dem Buch "A Million Random Digits with 100,000 Normal Deviates" \cite{amilli}
kommen. 

Es gibt zwei Arten wie die Challenge gewonnen werden kann.
Die eine Möglichkeit ist es, die Kolmogorov Komplexität zu verwenden und ein 
Programm zu schreiben, dass die ursprüngliche Datei erzeugt. 
Die größe des Programms muss kleiner sein, als die zu komprimierende Datei. 
Es geht dabei darum zu zeigen, dass die Kolmogorov Komplexität kleiner ist 
als die Größe der Datei selbst.

Die andere Möglichkeit ist es, ein System zu entwickeln, dass Dateien mit normalverteilten
Zufallszahlen komprimieren und vollständig wieder aus der komprimierten Datei 
dekomprimieren kann.
Die größe des Systems spielt dabei keine Rolle, weil das System mehr als nur eine 
Datei erfolgreich komprimieren und dekomprimieren muss.
Es ist bewiesen, dass dieser Ansatz unmöglich ist \cite{nelson}. 
Der Grund dafür ist, dass die Entropie der größe der Datei selbst entspricht und 
ohne Informationsverlust nicht verkleinert werden kann. 

Bis heute hat noch niemand die Challenge gewonnen. 
Das ist nicht verwunderlich, da
die zweite Möglichkeit zu gewinnen bewiesenermaßen unmöglich ist.
Hingegen ist die erste Möglichkeit mit dem Kolmogorov Komplexität nur 
vermutlich unmöglich.
Das bedeutet es könnte eine Lösung geben, ist aber nach jetzigem Stand unwahrscheinlich.

Die Challenge ist ein perfektes Beispiel dafür, wieso es Sinn macht die Daten zu 
kennen die man komprimieren will.
Sie zeigt die Limitationen der Datenkompression auf.


\section{Aufbau und Struktur von Bilddaten}

Dieser Abschnitt beschäftigt sich mit der dem Aufbau und der Struktur von Bilddaten.
Es gibt verschiedene Möglichkeiten Bilddaten zu speichern und darzustellen.
In dieser Arbeit werden die Bilder in Form einer Rastergrafik/ Bitmap gespeichert, dem
RGB Format.
Eine Rastergrafik ist eine pixelbasierte Darstellung eines Bildes.
Die Auflösung eines Bildes gibt an, wie viele Pixel in der Breite und Höhe vorhanden sind.
Ein Pixel ist die kleinste diskrete Einheit in einem digitalen Bild.

Das RGB Format ist eine Möglichkeit, Farbinformationen in digitalen Bildern zu 
repräsentieren und zu speichern. 
RGB steht für die Farben Rot (R), Grün (G) und Blau (B). 
Es ist ein additives Farbmodell und jede Farbe wird durch eine Kombination 
der drei Grundfarben erzeugt, auch weiß und schwarz. \cite{rite}

Ein Beispielbild mit einer Auflösung von 800 Pixel Breite und 800 Pixel Höhe wird in 
dem RGB Format durch eine Matrix gespeichert.
Die zugehörige Matrix hat eine Größe von (800, 800, 3), (Breite, Höhe, Farbkanäle).
Die dritte Dimension ist die Anzahl der Farbkanäle.
Für jeden Pixel werden drei Werte, die die Farbe des Pixels beschreiben, gespeichert.
Der erste Wert ist der Rotanteil, der zweite der Grünanteil und der dritte 
den Blauanteil.
Das RGB Format verwendet normalerweise 8 Bit pro Farbkanal, was eine Farbtiefe von 
24 Bit pro Pixel ergibt.
Jeder Farbwert eines Pixels hat 8 Bit, bzw. 1 Byte Speicher zur Verfügung und kann 
Werte zwischen 0 und 255 annehmen.

Bilder im RGB Format sind nach einem bestimmten Schema aufgebaut, weshalb 
Strukturen in den Bilddaten entstehen. 
Diese Strukturen können bei der Datenkompression helfen die Bilder zu komprimieren.
Eine Struktur im RGB Format ist die Darstellung von Nichtfarben 
wie schwarz und weiß. 
Um schwarz, bzw. weiß darzustellen muss ein Pixel für alle drei Farbkanälwerte entweder 
den Wert 0 oder 255 annehmen.
Um reine Farben darzustellen wie rot, wird nur ein Pixelwert für den Rotanteil 
benötigt, während der Grün- und Blauanteil auf Null gesetzt ist.
Ebenso gibt es viele Farbmischungen bei denen eine der Grundfarben nicht benötigt 
wird und somit deren Anteil im Pixel auf 0 gesetzt wird.
Diese Strukturen entstehen durch das Speichern des Bildes im RGB Format.
Eine weitere Struktur, die in den meisten Bildern vorhanden ist, ist, dass benachbarte 
Pixel meist ähnliche oder die gleiche Farbe besitzen.
Das hat damit zu tun, dass es in Bildern meistens Regionen gibt, die zu einem 
Objekt oder Bildteil gehören, die eine homogene Farbe besitzen.

Diese Strukturen und Wiederholungen in Bilddaten können ausgenutzt werden um die 
Daten zu komprimieren. 
Bei der Datenkompression ist es entscheidend, welche Algorithmen diese in Bildern
spezifisch enthaltenen Redundanzen erkennen und ausnutzen kann. 

% Todo vlt Vergleich mit Textdaten, Unterschiede!!!!, hier oder später aber wäre interessant


\section{Messbarkeit der Kompressionsalgorithmen}


- Messbarkeit definieren um Ergebnisse der Algorithmen zu vergleichen und zu entscheiden
was ist ein guter Kompressionsalgorithmus

Wie ist Kompressionsalgorithmus aufgebaut: 
    Daten die zu komp sind -> Kompression (Vorverarbeitung möglicherweise) 
    -> Dekompression (Vorverarbeitung der Daten rückgängig machen)
    -> Weil verlustfrei müssen nach komp und dekomp Originaldaten rauskommen

- Kompressionsrate, wie wird berechnet, Formel
- Kompressionsgeschwindigkeit, wie wird gemessen, möglicherweise Vorverarbeitungsschritte,
die zur Zeitmessung dazugehören
- Dekompressionsgeschwindigkeit, wie wird gemessen, aus komp Daten Unkomprimierte Daten

Abwiegen und Gewichten je nach Anwendung.
Bsp. echtzeit, nutzlos lange dekomp oder komp aber beste Kompressionsrate
Vlt. nur Dekomp entscheidend, Bsp. Inet.
Möglicherweise nur 1 interessant Komp oder Dekomp, Bsp.









\section{Introduction}
This document is a model and instructions for \LaTeX.
Please observe the conference page \cite{autor2023} limits.

\section{Ease of Use}

\subsection{Maintaining the Integrity of the Specifications}

The IEEEtran class file is used to format your paper and style the text. All margins,
column widths, line spaces, and text fonts are prescribed; please do not
alter them. You may note peculiarities. For example, the head margin
measures proportionately more than is customary. This measurement
and others are deliberate, using specifications that anticipate your paper
as one part of the entire proceedings, and not as an independent document.
Please do not revise any of the current designations.


\bibliographystyle{IEEEtran}
\bibliography{mybib}

\end{document}
